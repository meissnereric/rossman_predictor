
% Default to the notebook output style

    


% Inherit from the specified cell style.




    
\documentclass{article}

    
    
    \usepackage{graphicx} % Used to insert images
    \usepackage{adjustbox} % Used to constrain images to a maximum size 
    \usepackage{color} % Allow colors to be defined
    \usepackage{enumerate} % Needed for markdown enumerations to work
    \usepackage{geometry} % Used to adjust the document margins
    \usepackage{amsmath} % Equations
    \usepackage{amssymb} % Equations
    \usepackage{eurosym} % defines \euro
    \usepackage[mathletters]{ucs} % Extended unicode (utf-8) support
    \usepackage[utf8x]{inputenc} % Allow utf-8 characters in the tex document
    \usepackage{fancyvrb} % verbatim replacement that allows latex
    \usepackage{grffile} % extends the file name processing of package graphics 
                         % to support a larger range 
    % The hyperref package gives us a pdf with properly built
    % internal navigation ('pdf bookmarks' for the table of contents,
    % internal cross-reference links, web links for URLs, etc.)
    \usepackage{hyperref}
    \usepackage{longtable} % longtable support required by pandoc >1.10
    \usepackage{booktabs}  % table support for pandoc > 1.12.2
    

    
    
    \definecolor{orange}{cmyk}{0,0.4,0.8,0.2}
    \definecolor{darkorange}{rgb}{.71,0.21,0.01}
    \definecolor{darkgreen}{rgb}{.12,.54,.11}
    \definecolor{myteal}{rgb}{.26, .44, .56}
    \definecolor{gray}{gray}{0.45}
    \definecolor{lightgray}{gray}{.95}
    \definecolor{mediumgray}{gray}{.8}
    \definecolor{inputbackground}{rgb}{.95, .95, .85}
    \definecolor{outputbackground}{rgb}{.95, .95, .95}
    \definecolor{traceback}{rgb}{1, .95, .95}
    % ansi colors
    \definecolor{red}{rgb}{.6,0,0}
    \definecolor{green}{rgb}{0,.65,0}
    \definecolor{brown}{rgb}{0.6,0.6,0}
    \definecolor{blue}{rgb}{0,.145,.698}
    \definecolor{purple}{rgb}{.698,.145,.698}
    \definecolor{cyan}{rgb}{0,.698,.698}
    \definecolor{lightgray}{gray}{0.5}
    
    % bright ansi colors
    \definecolor{darkgray}{gray}{0.25}
    \definecolor{lightred}{rgb}{1.0,0.39,0.28}
    \definecolor{lightgreen}{rgb}{0.48,0.99,0.0}
    \definecolor{lightblue}{rgb}{0.53,0.81,0.92}
    \definecolor{lightpurple}{rgb}{0.87,0.63,0.87}
    \definecolor{lightcyan}{rgb}{0.5,1.0,0.83}
    
    % commands and environments needed by pandoc snippets
    % extracted from the output of `pandoc -s`
    \providecommand{\tightlist}{%
      \setlength{\itemsep}{0pt}\setlength{\parskip}{0pt}}
    \DefineVerbatimEnvironment{Highlighting}{Verbatim}{commandchars=\\\{\}}
    % Add ',fontsize=\small' for more characters per line
    \newenvironment{Shaded}{}{}
    \newcommand{\KeywordTok}[1]{\textcolor[rgb]{0.00,0.44,0.13}{\textbf{{#1}}}}
    \newcommand{\DataTypeTok}[1]{\textcolor[rgb]{0.56,0.13,0.00}{{#1}}}
    \newcommand{\DecValTok}[1]{\textcolor[rgb]{0.25,0.63,0.44}{{#1}}}
    \newcommand{\BaseNTok}[1]{\textcolor[rgb]{0.25,0.63,0.44}{{#1}}}
    \newcommand{\FloatTok}[1]{\textcolor[rgb]{0.25,0.63,0.44}{{#1}}}
    \newcommand{\CharTok}[1]{\textcolor[rgb]{0.25,0.44,0.63}{{#1}}}
    \newcommand{\StringTok}[1]{\textcolor[rgb]{0.25,0.44,0.63}{{#1}}}
    \newcommand{\CommentTok}[1]{\textcolor[rgb]{0.38,0.63,0.69}{\textit{{#1}}}}
    \newcommand{\OtherTok}[1]{\textcolor[rgb]{0.00,0.44,0.13}{{#1}}}
    \newcommand{\AlertTok}[1]{\textcolor[rgb]{1.00,0.00,0.00}{\textbf{{#1}}}}
    \newcommand{\FunctionTok}[1]{\textcolor[rgb]{0.02,0.16,0.49}{{#1}}}
    \newcommand{\RegionMarkerTok}[1]{{#1}}
    \newcommand{\ErrorTok}[1]{\textcolor[rgb]{1.00,0.00,0.00}{\textbf{{#1}}}}
    \newcommand{\NormalTok}[1]{{#1}}
    
    % Define a nice break command that doesn't care if a line doesn't already
    % exist.
    \def\br{\hspace*{\fill} \\* }
    % Math Jax compatability definitions
    \def\gt{>}
    \def\lt{<}
    % Document parameters
    \title{Data Science Mid Project Checkpoint}
    \author{Eric Meissner, Neal Smart, Maneesh Mohanavilasam}
    
    
    

    % Pygments definitions
    
\makeatletter
\def\PY@reset{\let\PY@it=\relax \let\PY@bf=\relax%
    \let\PY@ul=\relax \let\PY@tc=\relax%
    \let\PY@bc=\relax \let\PY@ff=\relax}
\def\PY@tok#1{\csname PY@tok@#1\endcsname}
\def\PY@toks#1+{\ifx\relax#1\empty\else%
    \PY@tok{#1}\expandafter\PY@toks\fi}
\def\PY@do#1{\PY@bc{\PY@tc{\PY@ul{%
    \PY@it{\PY@bf{\PY@ff{#1}}}}}}}
\def\PY#1#2{\PY@reset\PY@toks#1+\relax+\PY@do{#2}}

\expandafter\def\csname PY@tok@gd\endcsname{\def\PY@tc##1{\textcolor[rgb]{0.63,0.00,0.00}{##1}}}
\expandafter\def\csname PY@tok@gu\endcsname{\let\PY@bf=\textbf\def\PY@tc##1{\textcolor[rgb]{0.50,0.00,0.50}{##1}}}
\expandafter\def\csname PY@tok@gt\endcsname{\def\PY@tc##1{\textcolor[rgb]{0.00,0.27,0.87}{##1}}}
\expandafter\def\csname PY@tok@gs\endcsname{\let\PY@bf=\textbf}
\expandafter\def\csname PY@tok@gr\endcsname{\def\PY@tc##1{\textcolor[rgb]{1.00,0.00,0.00}{##1}}}
\expandafter\def\csname PY@tok@cm\endcsname{\let\PY@it=\textit\def\PY@tc##1{\textcolor[rgb]{0.25,0.50,0.50}{##1}}}
\expandafter\def\csname PY@tok@vg\endcsname{\def\PY@tc##1{\textcolor[rgb]{0.10,0.09,0.49}{##1}}}
\expandafter\def\csname PY@tok@m\endcsname{\def\PY@tc##1{\textcolor[rgb]{0.40,0.40,0.40}{##1}}}
\expandafter\def\csname PY@tok@mh\endcsname{\def\PY@tc##1{\textcolor[rgb]{0.40,0.40,0.40}{##1}}}
\expandafter\def\csname PY@tok@go\endcsname{\def\PY@tc##1{\textcolor[rgb]{0.53,0.53,0.53}{##1}}}
\expandafter\def\csname PY@tok@ge\endcsname{\let\PY@it=\textit}
\expandafter\def\csname PY@tok@vc\endcsname{\def\PY@tc##1{\textcolor[rgb]{0.10,0.09,0.49}{##1}}}
\expandafter\def\csname PY@tok@il\endcsname{\def\PY@tc##1{\textcolor[rgb]{0.40,0.40,0.40}{##1}}}
\expandafter\def\csname PY@tok@cs\endcsname{\let\PY@it=\textit\def\PY@tc##1{\textcolor[rgb]{0.25,0.50,0.50}{##1}}}
\expandafter\def\csname PY@tok@cp\endcsname{\def\PY@tc##1{\textcolor[rgb]{0.74,0.48,0.00}{##1}}}
\expandafter\def\csname PY@tok@gi\endcsname{\def\PY@tc##1{\textcolor[rgb]{0.00,0.63,0.00}{##1}}}
\expandafter\def\csname PY@tok@gh\endcsname{\let\PY@bf=\textbf\def\PY@tc##1{\textcolor[rgb]{0.00,0.00,0.50}{##1}}}
\expandafter\def\csname PY@tok@ni\endcsname{\let\PY@bf=\textbf\def\PY@tc##1{\textcolor[rgb]{0.60,0.60,0.60}{##1}}}
\expandafter\def\csname PY@tok@nl\endcsname{\def\PY@tc##1{\textcolor[rgb]{0.63,0.63,0.00}{##1}}}
\expandafter\def\csname PY@tok@nn\endcsname{\let\PY@bf=\textbf\def\PY@tc##1{\textcolor[rgb]{0.00,0.00,1.00}{##1}}}
\expandafter\def\csname PY@tok@no\endcsname{\def\PY@tc##1{\textcolor[rgb]{0.53,0.00,0.00}{##1}}}
\expandafter\def\csname PY@tok@na\endcsname{\def\PY@tc##1{\textcolor[rgb]{0.49,0.56,0.16}{##1}}}
\expandafter\def\csname PY@tok@nb\endcsname{\def\PY@tc##1{\textcolor[rgb]{0.00,0.50,0.00}{##1}}}
\expandafter\def\csname PY@tok@nc\endcsname{\let\PY@bf=\textbf\def\PY@tc##1{\textcolor[rgb]{0.00,0.00,1.00}{##1}}}
\expandafter\def\csname PY@tok@nd\endcsname{\def\PY@tc##1{\textcolor[rgb]{0.67,0.13,1.00}{##1}}}
\expandafter\def\csname PY@tok@ne\endcsname{\let\PY@bf=\textbf\def\PY@tc##1{\textcolor[rgb]{0.82,0.25,0.23}{##1}}}
\expandafter\def\csname PY@tok@nf\endcsname{\def\PY@tc##1{\textcolor[rgb]{0.00,0.00,1.00}{##1}}}
\expandafter\def\csname PY@tok@si\endcsname{\let\PY@bf=\textbf\def\PY@tc##1{\textcolor[rgb]{0.73,0.40,0.53}{##1}}}
\expandafter\def\csname PY@tok@s2\endcsname{\def\PY@tc##1{\textcolor[rgb]{0.73,0.13,0.13}{##1}}}
\expandafter\def\csname PY@tok@vi\endcsname{\def\PY@tc##1{\textcolor[rgb]{0.10,0.09,0.49}{##1}}}
\expandafter\def\csname PY@tok@nt\endcsname{\let\PY@bf=\textbf\def\PY@tc##1{\textcolor[rgb]{0.00,0.50,0.00}{##1}}}
\expandafter\def\csname PY@tok@nv\endcsname{\def\PY@tc##1{\textcolor[rgb]{0.10,0.09,0.49}{##1}}}
\expandafter\def\csname PY@tok@s1\endcsname{\def\PY@tc##1{\textcolor[rgb]{0.73,0.13,0.13}{##1}}}
\expandafter\def\csname PY@tok@kd\endcsname{\let\PY@bf=\textbf\def\PY@tc##1{\textcolor[rgb]{0.00,0.50,0.00}{##1}}}
\expandafter\def\csname PY@tok@sh\endcsname{\def\PY@tc##1{\textcolor[rgb]{0.73,0.13,0.13}{##1}}}
\expandafter\def\csname PY@tok@sc\endcsname{\def\PY@tc##1{\textcolor[rgb]{0.73,0.13,0.13}{##1}}}
\expandafter\def\csname PY@tok@sx\endcsname{\def\PY@tc##1{\textcolor[rgb]{0.00,0.50,0.00}{##1}}}
\expandafter\def\csname PY@tok@bp\endcsname{\def\PY@tc##1{\textcolor[rgb]{0.00,0.50,0.00}{##1}}}
\expandafter\def\csname PY@tok@c1\endcsname{\let\PY@it=\textit\def\PY@tc##1{\textcolor[rgb]{0.25,0.50,0.50}{##1}}}
\expandafter\def\csname PY@tok@kc\endcsname{\let\PY@bf=\textbf\def\PY@tc##1{\textcolor[rgb]{0.00,0.50,0.00}{##1}}}
\expandafter\def\csname PY@tok@c\endcsname{\let\PY@it=\textit\def\PY@tc##1{\textcolor[rgb]{0.25,0.50,0.50}{##1}}}
\expandafter\def\csname PY@tok@mf\endcsname{\def\PY@tc##1{\textcolor[rgb]{0.40,0.40,0.40}{##1}}}
\expandafter\def\csname PY@tok@err\endcsname{\def\PY@bc##1{\setlength{\fboxsep}{0pt}\fcolorbox[rgb]{1.00,0.00,0.00}{1,1,1}{\strut ##1}}}
\expandafter\def\csname PY@tok@mb\endcsname{\def\PY@tc##1{\textcolor[rgb]{0.40,0.40,0.40}{##1}}}
\expandafter\def\csname PY@tok@ss\endcsname{\def\PY@tc##1{\textcolor[rgb]{0.10,0.09,0.49}{##1}}}
\expandafter\def\csname PY@tok@sr\endcsname{\def\PY@tc##1{\textcolor[rgb]{0.73,0.40,0.53}{##1}}}
\expandafter\def\csname PY@tok@mo\endcsname{\def\PY@tc##1{\textcolor[rgb]{0.40,0.40,0.40}{##1}}}
\expandafter\def\csname PY@tok@kn\endcsname{\let\PY@bf=\textbf\def\PY@tc##1{\textcolor[rgb]{0.00,0.50,0.00}{##1}}}
\expandafter\def\csname PY@tok@mi\endcsname{\def\PY@tc##1{\textcolor[rgb]{0.40,0.40,0.40}{##1}}}
\expandafter\def\csname PY@tok@gp\endcsname{\let\PY@bf=\textbf\def\PY@tc##1{\textcolor[rgb]{0.00,0.00,0.50}{##1}}}
\expandafter\def\csname PY@tok@o\endcsname{\def\PY@tc##1{\textcolor[rgb]{0.40,0.40,0.40}{##1}}}
\expandafter\def\csname PY@tok@kr\endcsname{\let\PY@bf=\textbf\def\PY@tc##1{\textcolor[rgb]{0.00,0.50,0.00}{##1}}}
\expandafter\def\csname PY@tok@s\endcsname{\def\PY@tc##1{\textcolor[rgb]{0.73,0.13,0.13}{##1}}}
\expandafter\def\csname PY@tok@kp\endcsname{\def\PY@tc##1{\textcolor[rgb]{0.00,0.50,0.00}{##1}}}
\expandafter\def\csname PY@tok@w\endcsname{\def\PY@tc##1{\textcolor[rgb]{0.73,0.73,0.73}{##1}}}
\expandafter\def\csname PY@tok@kt\endcsname{\def\PY@tc##1{\textcolor[rgb]{0.69,0.00,0.25}{##1}}}
\expandafter\def\csname PY@tok@ow\endcsname{\let\PY@bf=\textbf\def\PY@tc##1{\textcolor[rgb]{0.67,0.13,1.00}{##1}}}
\expandafter\def\csname PY@tok@sb\endcsname{\def\PY@tc##1{\textcolor[rgb]{0.73,0.13,0.13}{##1}}}
\expandafter\def\csname PY@tok@k\endcsname{\let\PY@bf=\textbf\def\PY@tc##1{\textcolor[rgb]{0.00,0.50,0.00}{##1}}}
\expandafter\def\csname PY@tok@se\endcsname{\let\PY@bf=\textbf\def\PY@tc##1{\textcolor[rgb]{0.73,0.40,0.13}{##1}}}
\expandafter\def\csname PY@tok@sd\endcsname{\let\PY@it=\textit\def\PY@tc##1{\textcolor[rgb]{0.73,0.13,0.13}{##1}}}

\def\PYZbs{\char`\\}
\def\PYZus{\char`\_}
\def\PYZob{\char`\{}
\def\PYZcb{\char`\}}
\def\PYZca{\char`\^}
\def\PYZam{\char`\&}
\def\PYZlt{\char`\<}
\def\PYZgt{\char`\>}
\def\PYZsh{\char`\#}
\def\PYZpc{\char`\%}
\def\PYZdl{\char`\$}
\def\PYZhy{\char`\-}
\def\PYZsq{\char`\'}
\def\PYZdq{\char`\"}
\def\PYZti{\char`\~}
% for compatibility with earlier versions
\def\PYZat{@}
\def\PYZlb{[}
\def\PYZrb{]}
\makeatother


    % Exact colors from NB
    \definecolor{incolor}{rgb}{0.0, 0.0, 0.5}
    \definecolor{outcolor}{rgb}{0.545, 0.0, 0.0}



    
    % Prevent overflowing lines due to hard-to-break entities
    \sloppy 
    % Setup hyperref package
    \hypersetup{
      breaklinks=true,  % so long urls are correctly broken across lines
      colorlinks=true,
      urlcolor=blue,
      linkcolor=darkorange,
      citecolor=darkgreen,
      }
    % Slightly bigger margins than the latex defaults
    
    \geometry{verbose,tmargin=1in,bmargin=1in,lmargin=1in,rmargin=1in}
    
    

    \begin{document}
    
    
    \maketitle
    
    

    
    \section{Overview}\label{overview}

We are proposing to compete in the Kaggle competition setup by Rossman
Stores to predict store sales for the next 6 weeks of business from
previous business data.. All the information on the dataset is
\textcolor{blue}{\href{https://www.kaggle.com/c/rossmann-store-sales/data}{here}}.
There's information about competitors, daily sales grosses, etc.

\subsection{Code Location}\label{code-location}

Our exploratory code is found
\textcolor{blue}{\href{https://github.com/meissnereric/rossman_predictor}{on Eric's github}.}

\subsection{Descriptive Statistics and
Discussion}\label{descriptive-statistics-and-discussion}

    \begin{Verbatim}[commandchars=\\\{\}]
{\color{incolor}In [{\color{incolor}42}]:} \PY{k+kn}{import} \PY{n+nn}{numpy} \PY{k+kn}{as} \PY{n+nn}{np}
         \PY{k+kn}{import} \PY{n+nn}{pandas} \PY{k+kn}{as} \PY{n+nn}{pd}
         \PY{k+kn}{import} \PY{n+nn}{matplotlib.pyplot} \PY{k+kn}{as} \PY{n+nn}{plt}
         \PY{o}{\PYZpc{}}\PY{k}{matplotlib} inline
         \PY{o}{\PYZpc{}}\PY{k}{pwd}
\end{Verbatim}

            \begin{Verbatim}[commandchars=\\\{\}]
{\color{outcolor}Out[{\color{outcolor}42}]:} u'/home/eric/dev/projects/data\_science/rossman\_predictor'
\end{Verbatim}
        Our data came in three major files: train, test, and store. Training has the training data including daily sales and customers over a period of ~2 years (~1 million data points). , while test data has a period of 6 weeks for particular stores (not all of them, notably). Store has metadata about each store, which presumably affects the sales of a the stores. 

We first merge the store and training data together for ease of analysis, and display general information.

We also clean the StateHoliday field for string/int happiness. 
    \begin{Verbatim}[commandchars=\\\{\}]
{\color{incolor}In [{\color{incolor}43}]:} \PY{c}{\PYZsh{}Partial code taken from (https://www.kaggle.com/mmourafiq/rossmann\PYZhy{}store\PYZhy{}sales/data\PYZhy{}viz/notebook): }
         
         \PY{n}{train} \PY{o}{=} \PY{n}{pd}\PY{o}{.}\PY{n}{read\PYZus{}csv}\PY{p}{(}\PY{l+s}{\PYZsq{}}\PY{l+s}{data/train.csv}\PY{l+s}{\PYZsq{}}\PY{p}{)}
         \PY{c}{\PYZsh{}print(train[:5])}
         \PY{n}{store} \PY{o}{=} \PY{n}{pd}\PY{o}{.}\PY{n}{read\PYZus{}csv}\PY{p}{(}\PY{l+s}{\PYZsq{}}\PY{l+s}{data/store.csv}\PY{l+s}{\PYZsq{}}\PY{p}{)}
         \PY{c}{\PYZsh{}print(store[:5])}
         \PY{n}{all\PYZus{}data} \PY{o}{=} \PY{n}{pd}\PY{o}{.}\PY{n}{merge}\PY{p}{(}\PY{n}{train}\PY{p}{,} \PY{n}{store}\PY{p}{,} \PY{n}{on}\PY{o}{=}\PY{l+s}{\PYZsq{}}\PY{l+s}{Store}\PY{l+s}{\PYZsq{}}\PY{p}{,} \PY{n}{how}\PY{o}{=}\PY{l+s}{\PYZsq{}}\PY{l+s}{left}\PY{l+s}{\PYZsq{}}\PY{p}{)}
         \PY{n}{test} \PY{o}{=} \PY{n}{pd}\PY{o}{.}\PY{n}{read\PYZus{}csv}\PY{p}{(}\PY{l+s}{\PYZsq{}}\PY{l+s}{data/test.csv}\PY{l+s}{\PYZsq{}}\PY{p}{)}
         \PY{n}{all\PYZus{}data}\PY{p}{[}\PY{l+s}{\PYZsq{}}\PY{l+s}{StateHoliday}\PY{l+s}{\PYZsq{}}\PY{p}{]}\PY{p}{[}\PY{n}{all\PYZus{}data}\PY{p}{[}\PY{l+s}{\PYZsq{}}\PY{l+s}{StateHoliday}\PY{l+s}{\PYZsq{}}\PY{p}{]} \PY{o}{==} \PY{l+m+mi}{0} \PY{p}{]} \PY{o}{=} \PY{l+s}{\PYZsq{}}\PY{l+s}{0}\PY{l+s}{\PYZsq{}}
\end{Verbatim}

    \begin{Verbatim}[commandchars=\\\{\}]
/usr/local/lib/python2.7/dist-packages/ipykernel/\_\_main\_\_.py:9: SettingWithCopyWarning: 
A value is trying to be set on a copy of a slice from a DataFrame

See the the caveats in the documentation: http://pandas.pydata.org/pandas-docs/stable/indexing.html\#indexing-view-versus-copy
    \end{Verbatim}

    Basic stats of the data. Notice particularly the standard deviation is
high for both sales and customers, ipmlyign the data is highly variable
and reassuring us that a predictor is necessary.

	The assortment of goods that a store has is a major predictor of how well the store will do, as evidenced below. This is notable as the assortments as described on the competition are ordinal variables, but b does significantly better than a and c. A good feature to use to be sure.
    \begin{center}
    \adjustimage{max size={0.9\linewidth}{0.9\paperheight}}{AvgSalesVsMonthsByAssortments.png}
    \end{center}
    
    Unsurprisingly, a state holiday strongly affects sales.

    \begin{Verbatim}[commandchars=\\\{\}]
{\color{incolor}In [{\color{incolor}44}]:} \PY{n}{avg\PYZus{}stateholiday} \PY{o}{=} \PY{n}{all\PYZus{}data}\PY{p}{[}\PY{p}{[}\PY{l+s}{\PYZsq{}}\PY{l+s}{Sales}\PY{l+s}{\PYZsq{}}\PY{p}{,} \PY{l+s}{\PYZsq{}}\PY{l+s}{Customers}\PY{l+s}{\PYZsq{}}\PY{p}{,} \PY{l+s}{\PYZsq{}}\PY{l+s}{StateHoliday}\PY{l+s}{\PYZsq{}}\PY{p}{]}\PY{p}{]}\PY{o}{.}\PY{n}{groupby}\PY{p}{(}\PY{l+s}{\PYZsq{}}\PY{l+s}{StateHoliday}\PY{l+s}{\PYZsq{}}\PY{p}{)}\PY{o}{.}\PY{n}{mean}\PY{p}{(}\PY{p}{)}
         \PY{n}{avg\PYZus{}stateholiday}\PY{o}{.}\PY{n}{plot}\PY{p}{(}\PY{n}{kind}\PY{o}{=}\PY{l+s}{\PYZsq{}}\PY{l+s}{bar}\PY{l+s}{\PYZsq{}}\PY{p}{)}
\end{Verbatim}

            \begin{Verbatim}[commandchars=\\\{\}]
{\color{outcolor}Out[{\color{outcolor}44}]:} <matplotlib.axes.\_subplots.AxesSubplot at 0x7f950b8f25d0>
\end{Verbatim}
        
    \begin{center}
    \adjustimage{max size={0.8\linewidth}{0.8\paperheight}}{cleanData_files/cleanData_6_1.png}
    \end{center}
    { \hspace*{\fill} \\}
    
    Notably, having a promotion running for a particular day increases sales
by quite a large amount, while not increasing customers by nearly the
same rate. This implies people are spending more during those days,
instead of having simply higher rates of customers.

    \begin{Verbatim}[commandchars=\\\{\}]
{\color{incolor}In [{\color{incolor}45}]:} \PY{n}{avg\PYZus{}promotion} \PY{o}{=} \PY{n}{all\PYZus{}data}\PY{p}{[}\PY{p}{[}\PY{l+s}{\PYZsq{}}\PY{l+s}{Sales}\PY{l+s}{\PYZsq{}}\PY{p}{,} \PY{l+s}{\PYZsq{}}\PY{l+s}{Customers}\PY{l+s}{\PYZsq{}}\PY{p}{,} \PY{l+s}{\PYZsq{}}\PY{l+s}{Promo}\PY{l+s}{\PYZsq{}}\PY{p}{]}\PY{p}{]}\PY{o}{.}\PY{n}{groupby}\PY{p}{(}\PY{l+s}{\PYZsq{}}\PY{l+s}{Promo}\PY{l+s}{\PYZsq{}}\PY{p}{)}\PY{o}{.}\PY{n}{mean}\PY{p}{(}\PY{p}{)}
         \PY{n}{avg\PYZus{}promotion}\PY{o}{.}\PY{n}{plot}\PY{p}{(}\PY{n}{kind}\PY{o}{=}\PY{l+s}{\PYZsq{}}\PY{l+s}{bar}\PY{l+s}{\PYZsq{}}\PY{p}{)}
\end{Verbatim}

            \begin{Verbatim}[commandchars=\\\{\}]
{\color{outcolor}Out[{\color{outcolor}45}]:} <matplotlib.axes.\_subplots.AxesSubplot at 0x7f950b992ed0>
\end{Verbatim}
        
    \begin{center}
    \adjustimage{max size={0.8\linewidth}{0.8\paperheight}}{cleanData_files/cleanData_8_1.png}
    \end{center}
    { \hspace*{\fill} \\}
    
    Competition distances are clustered around having close competitors, and
in general it doesn't appear to affect sales strongly.

    \begin{Verbatim}[commandchars=\\\{\}]
{\color{incolor}In [{\color{incolor}46}]:} \PY{n}{all\PYZus{}data}\PY{o}{.}\PY{n}{hist}\PY{p}{(}\PY{l+s}{\PYZsq{}}\PY{l+s}{CompetitionDistance}\PY{l+s}{\PYZsq{}}\PY{p}{)}
\end{Verbatim}

            \begin{Verbatim}[commandchars=\\\{\}]
{\color{outcolor}Out[{\color{outcolor}46}]:} array([[<matplotlib.axes.\_subplots.AxesSubplot object at 0x7f950b5468d0>]], dtype=object)
\end{Verbatim}
        
    \begin{center}
    \adjustimage{max size={0.8\linewidth}{0.8\paperheight}}{cleanData_files/cleanData_10_1.png}
    \end{center}
    { \hspace*{\fill} \\}
    
    \begin{Verbatim}[commandchars=\\\{\}]
{\color{incolor}In [{\color{incolor}47}]:} \PY{c}{\PYZsh{} Bin the competition distance with 10 bins}
         \PY{n}{bins} \PY{o}{=} \PY{n}{np}\PY{o}{.}\PY{n}{linspace}\PY{p}{(}\PY{n}{all\PYZus{}data}\PY{p}{[}\PY{l+s}{\PYZsq{}}\PY{l+s}{CompetitionDistance}\PY{l+s}{\PYZsq{}}\PY{p}{]}\PY{o}{.}\PY{n}{min}\PY{p}{(}\PY{p}{)}\PY{p}{,} \PY{n}{all\PYZus{}data}\PY{o}{.}\PY{n}{CompetitionDistance}\PY{o}{.}\PY{n}{max}\PY{p}{(}\PY{p}{)}\PY{p}{,} \PY{l+m+mi}{10}\PY{p}{)}
         
         \PY{n}{competition\PYZus{}bins} \PY{o}{=} \PY{n}{all\PYZus{}data}\PY{p}{[}\PY{p}{[}\PY{l+s}{\PYZsq{}}\PY{l+s}{Sales}\PY{l+s}{\PYZsq{}}\PY{p}{,} \PY{l+s}{\PYZsq{}}\PY{l+s}{Customers}\PY{l+s}{\PYZsq{}}\PY{p}{]}\PY{p}{]}\PY{o}{.}\PY{n}{groupby}\PY{p}{(}\PY{n}{pd}\PY{o}{.}\PY{n}{cut}\PY{p}{(}\PY{n}{all\PYZus{}data}\PY{p}{[}\PY{l+s}{\PYZsq{}}\PY{l+s}{CompetitionDistance}\PY{l+s}{\PYZsq{}}\PY{p}{]}\PY{p}{,} \PY{n}{bins}\PY{p}{)}\PY{p}{)}
         \PY{n}{competition\PYZus{}avg} \PY{o}{=} \PY{n}{competition\PYZus{}bins}\PY{o}{.}\PY{n}{mean}\PY{p}{(}\PY{p}{)}
         \PY{n}{competition\PYZus{}avg}\PY{o}{.}\PY{n}{plot}\PY{p}{(}\PY{n}{kind}\PY{o}{=}\PY{l+s}{\PYZsq{}}\PY{l+s}{bar}\PY{l+s}{\PYZsq{}}\PY{p}{)}
\end{Verbatim}

            \begin{Verbatim}[commandchars=\\\{\}]
{\color{outcolor}Out[{\color{outcolor}47}]:} <matplotlib.axes.\_subplots.AxesSubplot at 0x7f950ba39950>
\end{Verbatim}
        
    \begin{center}
    \adjustimage{max size={0.8\linewidth}{0.8\paperheight}}{cleanData_files/cleanData_11_1.png}
    \end{center}
    { \hspace*{\fill} \\}
    
    Sundays have almost no sales.

Sales peak in July and December, the peak of summer and Christmas season
(The data is from Rossman stores, which are based in Germany so
Christmas would be a major holiday.)

    \begin{Verbatim}[commandchars=\\\{\}]
{\color{incolor}In [{\color{incolor}49}]:} \PY{c}{\PYZsh{}Done By Neal}
         \PY{n}{train}\PY{p}{[}\PY{l+s}{\PYZsq{}}\PY{l+s}{Date}\PY{l+s}{\PYZsq{}}\PY{p}{]} \PY{o}{=} \PY{n}{pd}\PY{o}{.}\PY{n}{to\PYZus{}datetime}\PY{p}{(}\PY{n}{train}\PY{p}{[}\PY{l+s}{\PYZsq{}}\PY{l+s}{Date}\PY{l+s}{\PYZsq{}}\PY{p}{]}\PY{p}{)}
         \PY{n}{train}\PY{p}{[}\PY{p}{:}\PY{l+m+mi}{5}\PY{p}{]}\PY{p}{[}\PY{l+s}{\PYZsq{}}\PY{l+s}{Date}\PY{l+s}{\PYZsq{}}\PY{p}{]}\PY{o}{.}\PY{n}{dt}\PY{o}{.}\PY{n}{dayofweek}
         \PY{n}{train}\PY{p}{[}\PY{l+s}{\PYZsq{}}\PY{l+s}{DayOfWeek}\PY{l+s}{\PYZsq{}}\PY{p}{]} \PY{o}{=} \PY{n}{train}\PY{p}{[}\PY{l+s}{\PYZsq{}}\PY{l+s}{Date}\PY{l+s}{\PYZsq{}}\PY{p}{]}\PY{o}{.}\PY{n}{dt}\PY{o}{.}\PY{n}{dayofweek}
         \PY{n}{train}\PY{p}{[}\PY{l+s}{\PYZsq{}}\PY{l+s}{Month}\PY{l+s}{\PYZsq{}}\PY{p}{]} \PY{o}{=} \PY{n}{train}\PY{p}{[}\PY{l+s}{\PYZsq{}}\PY{l+s}{Date}\PY{l+s}{\PYZsq{}}\PY{p}{]}\PY{o}{.}\PY{n}{dt}\PY{o}{.}\PY{n}{month}
         \PY{n}{train}\PY{p}{[}\PY{l+s}{\PYZsq{}}\PY{l+s}{Year}\PY{l+s}{\PYZsq{}}\PY{p}{]} \PY{o}{=} \PY{n}{train}\PY{p}{[}\PY{l+s}{\PYZsq{}}\PY{l+s}{Date}\PY{l+s}{\PYZsq{}}\PY{p}{]}\PY{o}{.}\PY{n}{dt}\PY{o}{.}\PY{n}{year}
         \PY{n}{avg\PYZus{}month} \PY{o}{=} \PY{n}{train}\PY{p}{[}\PY{p}{[}\PY{l+s}{\PYZsq{}}\PY{l+s}{Sales}\PY{l+s}{\PYZsq{}}\PY{p}{,} \PY{l+s}{\PYZsq{}}\PY{l+s}{Month}\PY{l+s}{\PYZsq{}}\PY{p}{]}\PY{p}{]}\PY{o}{.}\PY{n}{groupby}\PY{p}{(}\PY{l+s}{\PYZsq{}}\PY{l+s}{Month}\PY{l+s}{\PYZsq{}}\PY{p}{)}\PY{o}{.}\PY{n}{mean}\PY{p}{(}\PY{p}{)}
         \PY{n}{avg\PYZus{}month}\PY{o}{.}\PY{n}{plot}\PY{p}{(}\PY{n}{kind}\PY{o}{=}\PY{l+s}{\PYZsq{}}\PY{l+s}{bar}\PY{l+s}{\PYZsq{}}\PY{p}{)}
         \PY{n}{avg\PYZus{}day} \PY{o}{=} \PY{n}{train}\PY{p}{[}\PY{p}{[}\PY{l+s}{\PYZsq{}}\PY{l+s}{Sales}\PY{l+s}{\PYZsq{}}\PY{p}{,} \PY{l+s}{\PYZsq{}}\PY{l+s}{DayOfWeek}\PY{l+s}{\PYZsq{}}\PY{p}{]}\PY{p}{]}\PY{o}{.}\PY{n}{groupby}\PY{p}{(}\PY{l+s}{\PYZsq{}}\PY{l+s}{DayOfWeek}\PY{l+s}{\PYZsq{}}\PY{p}{)}\PY{o}{.}\PY{n}{mean}\PY{p}{(}\PY{p}{)}
         \PY{n}{avg\PYZus{}day}\PY{o}{.}\PY{n}{plot}\PY{p}{(}\PY{n}{kind}\PY{o}{=}\PY{l+s}{\PYZsq{}}\PY{l+s}{bar}\PY{l+s}{\PYZsq{}}\PY{p}{)}
         \PY{c}{\PYZsh{}sales by day of week}
         \PY{n}{sale\PYZus{}dayofweek} \PY{o}{=} \PY{n}{pd}\PY{o}{.}\PY{n}{pivot\PYZus{}table}\PY{p}{(}\PY{n}{train}\PY{p}{,} \PY{n}{values}\PY{o}{=}\PY{l+s}{\PYZsq{}}\PY{l+s}{Sales}\PY{l+s}{\PYZsq{}}\PY{p}{,} \PY{n}{index}\PY{o}{=}\PY{p}{[}\PY{l+s}{\PYZsq{}}\PY{l+s}{Year}\PY{l+s}{\PYZsq{}}\PY{p}{,}\PY{l+s}{\PYZsq{}}\PY{l+s}{Store}\PY{l+s}{\PYZsq{}}\PY{p}{]}\PY{p}{,} \PY{n}{columns}\PY{o}{=}\PY{p}{[}\PY{l+s}{\PYZsq{}}\PY{l+s}{DayOfWeek}\PY{l+s}{\PYZsq{}}\PY{p}{]}\PY{p}{)}
         
         \PY{c}{\PYZsh{}sales by month}
         \PY{n}{sale\PYZus{}month} \PY{o}{=} \PY{n}{pd}\PY{o}{.}\PY{n}{pivot\PYZus{}table}\PY{p}{(}\PY{n}{train}\PY{p}{,} \PY{n}{values}\PY{o}{=}\PY{l+s}{\PYZsq{}}\PY{l+s}{Sales}\PY{l+s}{\PYZsq{}}\PY{p}{,} \PY{n}{index}\PY{o}{=}\PY{p}{[}\PY{l+s}{\PYZsq{}}\PY{l+s}{Year}\PY{l+s}{\PYZsq{}}\PY{p}{,}\PY{l+s}{\PYZsq{}}\PY{l+s}{Store}\PY{l+s}{\PYZsq{}}\PY{p}{]}\PY{p}{,} \PY{n}{columns}\PY{o}{=}\PY{p}{[}\PY{l+s}{\PYZsq{}}\PY{l+s}{Month}\PY{l+s}{\PYZsq{}}\PY{p}{]}\PY{p}{)}
         \PY{c}{\PYZsh{}sale\PYZus{}dayofweek.plot(kind=\PYZsq{}box\PYZsq{})}
         \PY{c}{\PYZsh{}sale\PYZus{}month.plot(kind=\PYZsq{}box\PYZsq{})}
\end{Verbatim}

    \begin{center}
    \adjustimage{max size={0.8\linewidth}{0.8\paperheight}}{cleanData_files/cleanData_13_0.png}
    \end{center}
    { \hspace*{\fill} \\}
    
    \begin{center}
    \adjustimage{max size={0.8\linewidth}{0.8\paperheight}}{cleanData_files/cleanData_13_1.png}
    \end{center}
    { \hspace*{\fill} \\}
    
    \subsection{Place in Data Science
Model}\label{place-in-data-science-model}

Our model will use a typical data science procedure for batch data
analysis including a cleaning phase that feeds into a predictive
regression model for sales prediction.

\subsection{Next Steps}\label{next-steps}

The next steps mainly include deciding on a particular predictive mode,
through both empirical testing and looking at what popular methods for
time series regression are used in the field.

Because our goal is to build a sales predictor, visualization of the
results isn't a particular goal, though we plan to visualize and discuss
major patterns in the data that our analysis finds, such as having a
store with a promotion increasing sales by X or discussing the relation
of stores to their competitors. We will also include a visualization of
our predictor if relevant.

    \begin{Verbatim}[commandchars=\\\{\}]
{\color{incolor}In [{\color{incolor} }]:} 
\end{Verbatim}


    % Add a bibliography block to the postdoc
    
    
    
    \end{document}
